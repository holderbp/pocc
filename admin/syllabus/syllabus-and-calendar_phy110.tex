\documentclass{article}

\usepackage[margin=0.75in]{geometry}
\usepackage{enumerate}

\renewcommand{\labelitemii}{$\bullet$}


\begin{document}

\begin{center}
{\bf \Large
PHY 110 --- The Physics of Climate Change\\[6pt]
Fall 2025
}
\end{center}

\vspace{0.5cm}
\begin{center}
\begin{tabbing}
\hspace{2cm} \= Meeting Times \& Places \hspace{2cm} \={\bf Lecture:} \hspace{0.5cm} \=Tue, 10--10:50am, PAD 168\\
\>\>{\bf Laboratory:} :\>Wed, 9--10:50am, PAD 259\\
\>\>{\bf Discussion:}\>Thu, 10--10:50am, PAD 168\\
\\
\> Instructor: \> Ben Holder\\
\> Email (best method of contact): \> {\tt holderb@gvsu.edu}\\
\> Office: \> 152 PAD\\
\> Office Hours: \> Tu 11--12; W 2--3; Th 11-12 (PHC); and by appointment\\
\\
\> Course Website and Documents: \> Blackboard
\end{tabbing}
\end{center}

\vspace{0.5cm}
\begin{itemize}
\item[] {\bf Course Description}: Introduction to the physics of climate change and climate modeling. Introduces the physical processes that determine the Earth's temperature and its variation due to long-time-scale natural effects and contemporary anthropogenic influences. Topics include energy/equilibrium, blackbody radiation, the greenhouse effect, heat transfer by convection, and climate change timescales.

\item[] {\bf Credits}: 4

\item[]{\bf Prerequisites}: MTH 110 (Algebra)

\item[] {\bf Course Organization}: Each week we will consider a different climate-related topic and physical principles related to its analysis. Tuesday's class will generally be in a lecture format, with some assigned reading (from the textbook and/or lecture notes) expected ahead of time. Brief reading quizzes will assess class preparation. This lecture is to prepare students to complete the Laboratory (Wednesday) and Discussion (Thursday) assignments. Lab assignments should be mostly completed within the lab time period, but will be due on Friday.  Discussion assignments will be the homework assignment for that week and will be due the following Tuesday.  %\\[-6pt]


\item[]{\bf Textbook}: {\em Introduction to Modern Climate Change}, Andrew Dessler (3rd Ed, Cambridge, 2021)

    
\item[] {\bf Evaluation}: Grades will be determined from weekly laboratory and discussion/homework assignments, reading quizzes, and a final project/poster, under the following weighting:
    %
    \begin{center}
    \begin{tabular}{cc}
    Class Preparation/Participation & 10\%\\
    Discussion/Homework & 40\%\\
    Laboratory & 30\% \\
    Final Project & 20\%
    \end{tabular}
    \end{center}
    %
The lowest laboratory assignment will be dropped (allowing for an unexcused absence), other absences must be documented. Assignment grades will be given on a 5-point scale (e.g., with 100 being an ``A+'', 80 being a ``B-'', etc). The completion of an assignment will receive at least a ``C'' (75) and a demonstrated effort toward the correct answers will recieve at least a ``B''. Final grade boundaries are standard: A 93; A-  90; B+ 88; B 83; B- 80; C+ 78; C 73; C- 70; D+ 68; D 60.

\item[] {\bf Discussion/Homework}:  We will spend Thursdays working on the Discussion assignment in small groups.  The completion of these assignments after class will the Homework assignment for that week.  Students are encouraged to work together on these assignments, but you are expected to turn in your own original and idiosyncratic version of this work (I want to see your thought process in the responses). Late homework will not be accepted, but the lowest homework assignment grade will be dropped.

\thispagestyle{empty}
\newpage

\item[] {\bf Laboratory}: You will select a topic to investigate more deeply.  My primary goal here is that you {\em perform some analysis of a biological system, using mathematical tools and/or physical theories/techniques}. Thus, you should think of this as a (small) research project in which you perform some active calculation, derivation, model-fitting, etc; rather than learning and presenting a high-level description of a system. The process should probably include most of the following steps: (1) clearly identifying the biological system and describing its features quantitatively; (2) clarifying the physics/math needed to model the biological system; (3) gathering some data (either from the literature or your own experiment); (4)  performing some calculation/simulation/model-fitting. But, you are free to develop any idea that interests you, and I will be actively involved in helping to refine and execute your project. More information will be provided later in the semester. The project will culminate in a research report ($\sim10$ pages) and an oral presentation (10--15 minute). The paper will be due on the last day of classes and there will be intermediate deadlines for topic approval and abstract submission. 

\item[] {\bf Final Project}: The final week of the class will be devoted to the completion of small project that will be take the form of a poster presentation. These presentations will occur during our final exam period.  The final poster will be due at the final exam period, with an earlier deadline to propose and receive approval for your topic (I will help you to narrow its scope).

\item[] {\bf Final Exam Period}: There will be no final exam, but we will meet during the exam period to complete student presentations.


\item[] {\bf Course Objectives} (From the Syllabus of Record): After successful completion of the course, students will be able to\ldots
	%
	\begin{itemize}
	\item Describe the temperature and carbon dioxide time series over geological and contemporary timescales. 
	\item Explain how blackbody radiation of the Sun and albedo of the Earth determine the average incident solar energy flux. 
	\item Explain how physical parameters impact climate model predictions. 
	\item Analyze the effect of carbon dioxide (and other gases) on incident radiation of varying frequencies. 
	\item Experiment with objects and gases, demonstrating transparency/opaqueness to incident radiation of varying frequencies. 
	\item Calculate the equilibrium temperature of the earth with and without an atmosphere, and then including the effects of convection. 
	\item Estimate the uncertainties in measurements, e.g., of temperature and of proxy data, and in climate model output. 
	\item Describe the scientific conclusions presented in technical reports (e.g., by the Intergovernmental Panel on Climate Change (IPCC)) and media articles about climate change.
	\item Illustrate how evaporation and condensation of water occurs, and its dependence on temperature and humidity. 
	\item Build simple models of atmospheric energy transfer, leading to situations of equilibrium and/or "forcings" from out-of-equilibrium situations.
	\end{itemize}


\item[] {\bf Academic Honesty}: Academic Integrity is discussed in Section 223 of the {\em Student Code}. You are expected to complete the exams without unauthorized assistance and you should not provide assistance another student. Academic dishonesty will automatically result in an F for the assignment (for all parties involved) and will reported to the appropriate university authorities.  Flagrant violations of academic honesty may result in more severe penalties as determined by the appropriate university authorities.  Discussing an exam with a student who has not yet taken it is considered academic dishonesty.

\item[] {\bf Disabilities}: Any student who has special needs because of a learning, physical, or other disability should contact {\em Disability Support Resources} (DSR) at 616-331-2490.  If you have a disability and think that you will need assistance evacuating this classroom and/or building in an emergency, please make me aware so that the University can develop a plan with you to assist you.

\item[] {\bf Inclusion and Equity}: The campus of GVSU, including this classroom, is a safe and welcoming space for all students, regardless of age, gender, race, ethnic background, religious affiliation, sexual orientation, and gender identity. Please treat your fellow students with respect and fairness.

\end{itemize}

\pagestyle{empty}
\newpage

\begin{center}
{\Large Physics of Climate Change --- Fall 2025 --- Tentative Course Calendar}
\end{center}

\renewcommand{\arraystretch}{1.5}
\hspace{-1.7cm}%
\begin{tabular}{|l|c|c|c|}
\hline
{\bf Week}  & {\bf Lecture Topic (Tue)} & {\bf Laboratory (Wed)} &  {\bf Discussion (Thu)} \\
\hline
\hline
Aug 26--28 & Temperatures and C${\rm O}_2$ over Time & Measurement and Uncertainty & Earth's History of Temp \\
\hline
Sep 2--4 & Contemporary Climate Change & Global Average Temperature & Weather Indicators of Clim Ch \\
\hline
Sep 9--11 & Energy and Equilibrium & Mechanical equilibrium (leaky buckets I)& Energy and Energy Transf \\
\hline
Sep 16--18 & EM Radiation (light) and Blackbody & All the light you cannot see (BB rad) & BBody Rad \& Earth Temp \\
\hline
Sep 23--25  & Atmosphere and Greenhouse Eff I &  Heating by radiation & Earth temp:  w/Atmosphere \\
\hline
Sep 30 -- 2 & Atmosphere and Greenhouse Eff II & Absorption of radiation / Blanketing & Atmosph Absorption Spectra \\
\hline
Oct 7--9 & Water and Phase Transitions I &  Cloud Formation & Evaporation/Condensation \\
\hline
Oct 14--16 & Water and Phase Transitions II & Phase Transitions & Earth Temp: Rad-Conv Equil \\
\hline
Oct 21--23  & {\bf Fall Break (no class)} &  {\bf No Lab (probably)} & Intro to Milankovitch Cycles \\
\hline
Oct 28--30 & Paleo Perturbations of Temperature & Milankovich Cycles & More Paleo and Milankovitch \\
\hline
Nov 4--6 & Climate Sensitivity & IR absorption by GH gases & C${\rm O}_2$ absorption line \\
\hline
Nov 11--13 & Timescales of Climate Change & Equilibration: buckets and radiation & Rates of Change, Timescales \\
\hline
Nov 18--20  & Modeling Climate Change & \multicolumn{2}{|c|}{Computational Climate Models} \\
\hline
Nov 25--27 & Oceans and Ice & \multicolumn{2}{|c|}{\bf Thanksgiving (no class)}\\
\hline
Dec 2--4  &  {\bf work on project}  & Ice and Thermal Exp of ${\rm H}_2$O  & {\bf work on project} \\
\hline
\multicolumn{4}{|c|}{\bf FINAL EXAM TIMESLOT: Thursday, December 11, 2025, 10am--Noon (Poster Presentations)}\\
\hline
\end{tabular}


\pagestyle{empty}
\end{document}
